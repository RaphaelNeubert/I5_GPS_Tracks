\documentclass{article}

\usepackage[margin=3cm]{geometry}
\usepackage{amsmath}
\usepackage{amsfonts}
\usepackage{amssymb}
\usepackage{amscd}
\usepackage{standalone}
\usepackage{float}
\usepackage{color}
\usepackage[shortlabels]{enumitem}
\usepackage{graphicx}
\usepackage{caption}
\usepackage[ngerman]{babel}
\usepackage{lscape}
\usepackage{cancel}
\usepackage{dirtytalk}

\graphicspath{ {./images/} }

\begin{document}
\begin{titlepage}
    \centering
    {\scshape\LARGE Hochschule für Technik und Wirtschaft Dresden \par}
    \vspace{1cm}
    {\scshape\Large Softwaresystem \glqq GPS-Track-App\grqq\par}
    \vspace{1.5cm}
    {\huge\bfseries Prjektbericht\par}
    \vspace{2cm}
    {\Large\itshape Raphael Neubert, Alex Schechtel, Aleksandr Pronin, Quang Duy Pham, Tom Nicolai\par}
    \vfill

    {\large \today\par}
\end{titlepage}
\tableofcontents
\newpage
\begin{table}[H]
    \begin{tabular}{lllll}
    \textbf{Vorname} & \textbf{Nachname} & \textbf{Kürzel} &  &  \\
    Aleksandr        & Pronin            & AP              &  &  \\
    Alex             & Schechtel         & AS              &  &  \\
                     &                   &                 &  & 
    \end{tabular}
\end{table}
\section{Projektplanung}
\subsection{Aufgabenstellung}
    Im Rahmen unserer Projektarbeit im Modul \say{Software Engineering} hat Prof. Mario Neugebauer uns die Möglichkeit gegeben,
    eine Smartphone App zu entwickeln, welche das erstellen von GPS-Tracks für einen mobilen 
    Roboter ermöglicht. Dies hat den Hintergrund, dass die meisten GPS-Tack Apps sehr unflexibel sind und es daher nicht 
    möglich ist, Strecken auf Gegebenheiten des Roboters anzupassen. Das Hauptziel unserer Projektes war es also, eine App zu entwickeln, 
    welche Strecken des Roboters aufnimmt, die er optimal verarbeiten und umsetzen kann.
    Dabei mussten wir für jede implementierte Funktionalität,
    eine angenehme und einfach zu verstehende Darstellungsform für die App finden.
    Es sollte möglich sein die aufgenommenen GPS-Tracks abzuspeichern und in der App aufzulisten.
    Außerdem sollte man auf einem bereitgestellten Server die GPS-Tracks hochladen oder auch herunterladen können.
    Auf den Server sollte jeder Nutzer zugreifen können, welcher eine VPN-Verbindung zu der HTW-Dresden hat.
    Ebenso sollte es die Option geben, die GPS-Tracks nach dem abspeichern noch zu bearbeiten und dabei spezielle Punkte (special points of interests) 
    hinzuzufügen, sowie auch Punkte des aufegnommenen Weges zu verändern/verschieben.
    Bei der Aufnahme eines GPS-Tracks war es außerdem wichtig, die Genauigkeit des GPS-Signals so hoch wie möglich umzusetzen. 
\subsection{Auftraggeber}
    Die Rolle des Product Owners und Auftraggebers hat Herr Prof. Mario Neugebauer. Die Lehrgebiete von Prof. Neugebauer
    reichen von Programmier- und Datenbankgrundlagen bis hin zu mobilen Systemen und Kommunikationstechnik in 
    mobilen Netzen. Als Product Owner stand er in regelmäßigem Kontakt mit unserem Team, unterstützte uns und
    bewertete die Leistung unseres Teams am Ende der Iteration. Er verfügt über genügend Erfahrung und Wissen,
    um als technischer Ansprechpartner für das Team zu fungieren. Die Zusammenarbeit mit ihm war ein entscheidender Teil 
    unseres Projekts, und sie hat sich als sehr erfolgreich erwiesen.
\section{Ausgangssituation}
\section{Projektorganisation}
\newpage
\subsection{Kommunikation}
Durch die Pandemie bedingten Einschränkungen haben wir in \say{Softwareengineering-I} ausschließlich online kommuniziert. Später, in \say{Softwareengineering-II} war dies nicht mehr nötig. Da unser Team aus drei verschiedenen
Studiengängen und verschiedenen Jahrgängen bestand, war es durch die verschiedenen Stundenpläne und Nebenjobs 
schwer einen Termin zu finden an dem alle an der HTW sind bzw. an die HTW kommen konnten. Da die online Kommunikation 
schon bei Softwareengineering-I sehr gut funktionierte, entschlossen wir uns bei Softwareengineering-II weiterhin online 
zu kommunizieren.\par
\medskip
Für die online-Kommunikation im Team verwendeten wir die Dienst \say{Element}. Element ähnelt vom Grundaufbau dem 
Dienst \say{Discord}, unterscheidet sich aber insofern, als das Element Open-Source und dezentralisiert ist.
In unserem Element-\say{Space} erstellten wir uns zunächst zwei Kanäle. Einen für allgemeine Besprechungen, 
einen für  Programmierungs- bzw. Implementierungsspezifische Konversationen. Später im Projekt merkten wir, das 
es oft vorkam, das wichtige Informationen im Chat in unwichtigere Informationen untergingen. Wir erstellten
daher noch einen dritten Kanal für alle wichtigen links und Termine.
Für die Meetings verwendeten wir zunächst die von Element bereitgestellte Gruppenanrufs-Funktion.
Nach einigen Updates von Element gab es allerdings einige Probleme bei den Gruppenanrufen. Wir entschlossen 
uns daher von nun an für unsere Meetings die Plattform \say{Jitsi} zu verwenden.
Jitsi ist eine Open-Source Konferenzservice mit dem man im Browser, 
ohne Anmeldung mit zwei Klicks eine Konferenz starten kann. Unsere Erfahrungen mit Jitsi waren ausschließlich positiv.\par
\medskip
Um eine stetige Kommunikation zu gewährleisten und somit die Chance für das Eintreten bzw. die Folgen
von fehlerhafte Kommunikation zu minimieren, trafen wir uns am Anfang des Projektes wöchentlich. Am Ende 
des Projektes war unsere Aufgabenverteilung und unser gegenseitiges Vertrauen gut genug, sodass wir, solange
eine stetige Kommunikation auch asynchron gewährleistet war, zwischen Meetings auch mal zwei Wochen Platz lassen konnten.\par
\medskip
Während unsere Meetings zunächst sehr unkoordiniert abliefen, ergab sich mit der Zeit eine immer besser werdende 
Meetingstruktur. Anfangs haben wir für jedes Meeting sehr aufwändige Protokolle geschrieben. Nach einer Weile stellten 
wir fest das kaum jemand die Protokolle las und uns daher keinen Mehrwert gab. Um einen Ausgleich zwischen 
investierter Zeit und Mehrwert zu bekommen, entschieden wir uns, nur noch die Treffen mit dem Themensteller oder dem Coach
zu protokollieren und in dem Fall, dass jemand an einem Meeting nicht teilnehmen kann, wir uns im Nachhinein mit der 
Person in Verbindung setzen und ihr kurz das Meeting zusammenfassen. Eine weitere Verbesserung erzielten wir dadurch
das wir anfingen das jeder am Anfang des Meetings kurz zeigte, was er in der letzten Zeit gemacht hatte.
Dies wirkte sich auf den allgemeinen Überblick des Teams, aber auch auf die Motivation der einzelnen Personen positiv aus.
Es gab noch kleine weitere Änderungen, die sich positiv auf die Meetings ausgewirkt haben. Neben den 
durch Änderungen ausgelösten diskreten Verbesserungen war aber auch eine, durch den Zuwachs an allgemeiner 
Erfahrung und Kompetenz ausgelöste stetige Verbesserung bemerkbar. Die Früchte der Verbesserungen waren unter anderen 
ein besseren Überblick den jedes Teammitglied über das Projekt hatte, jeder wusste nach dem Meeting genau, was er 
zu erledigen hatte, die Meeting Zeit wurde effizienter genutzt so das in weniger Zeit
mehr Informationen ausgetauscht werden konnten.
\par
\medskip

Am Ende unterteilten sich die Meetings meistens in vier Teile. Als Erstes haben wir immer über die in Ergebnisse
seit dem letzten Meeting gesprochen. Wie bereits erwähnt geschah dies oft so, das die einzelnen Teilnehmer 
kurz ihre fortschritte erklärten bzw. vorführten.  Danach besprachen wir die grobe weitere Planung. Dazu zählte die 
Überprüfung ob wir uns im Zeitplan befinden, die Priorisierung bzw. das Umpriorisieren von Funktionalitäten und die 
Planung von Meetings mit dem Themensteller oder Coach. Als Drittes setzten wir die Planung der aktuellen Funktionalitäten 
fort und erstellten Issues für die Weiterentwicklung. Im letzten Teil des Meetings teilten wir dann die Issues 
den Teammitgliedern zu. Dabei konnte es auch vorkommen, dass wir niedrig priorisierte Issues niemanden zuteilten, sondern 
diese, falls jemand schon schneller als erwartet mit seinen Issues fertig war, dieser Person zugewiesen wurde.\par
\medskip
Um Zeitverschwendung zu vermeiden planten wir die Meetings mit dem Themensteller oder dem Coach nur nach Bedarf.
Wir schrieben uns unsere fragen immer auf und sobald genug zusammengekommen waren bzw. eine Frage sehr 
wichtig war, planten wir ein Meeting. Für die Besprechung von kleinen, wichtigen Entscheidungen,
die schnell getroffen werden mussten war oft auch eine kurze Mail an den Themensteller ausreichend.
Während wir bei Softwareengineering-I häufig Meetings mit unserem Coach hatten, reduzierte sich die Anzahl
bei Softwareengineering-II stark. Dies war vor allem der Tatsache geschuldet das wir uns mehr auf die Entwicklung
der eigentlichen Software konzentrierten aber auch der Tatsache das wir
in unserem arbeiten viel sicherer geworden sind und dadurch weniger Fragen auftraten.
Die Meetings sowohl mit dem Themensteller als auch mit dem Coach empfanden wir immer als sehr sinnvoll. Meist 
trafen wir uns direkt nach dem Meeting, um die vorgeschlagenen dinge in Issues zu überführen die wir dann später 
umsetzten.\par 
\medskip 
Perfekt waren unsere Meetings am Ende aber noch lange nicht. Ein Problem was wir zum Beispiel nur vermindern aber 
noch nicht komplett lösen konnten war das, sich im Meeting beschränken auf Konversationen, die möglichst alle betreffen.
Zum Beispiel haben wir manchmal sehr detailliert über die Serversynchronisation gesprochen und das, obwohl an der 
Entwicklung der Synchronisation nur zwei Personen beteiligt waren. Solche Situationen entstanden unbewusst und 
wirkte sich, für alle am detaillierten Thema unbeteiligten negativ auf die Qualität des Meetings aus.

\section{Dokumentation}
\section{Projektdurchführung}
\section{Ergebnisse}
Als Produkt unseres Projekts haben wir ein System entwickelt, das aus einer Android-App und einem Server für den Datenaustausch besteht.
Die Anwendung "GPS Tracks App" ermöglicht Nutzern mit einem Android-Smartphone, einen GPS-Wegpunkt oder eine Reihe von Wegpunkten auszuwählen 
und einen Roboter anweisen können, sich autonom zwischen den Punkten zu bewegen, wobei er bestimmte Aktionen (oder auch Script) ausführen kann. 
In der App haben wir Funktionen implementiert, um GPS-Routen zu erstellen, anzupassen, mit anderen Geräten über einen Server auszutauschen
und Routen zu löschen. Die App unterstützt auch die Funktion, eine bestimmte Aktion an einem GPS-Punkt zuspeichern. Die App kann auch in anderen 
Projekten eingesetzt werden. So kann man beispielsweise interessante Routen mit Freunden teilen und spezielle Hinweisen zu Wegpunkten anlegen.
Die App an sich ist gut durchdacht und verständlich, wenn man kurz eingearbeitet wurde.
\subsection{Reflexion der Teammitglieder}
\subsubsection{Tom Nicolai}
    Ich war mit im Bereich der Programmierung der App zuständig. In den Meetings haben wir uns immer neue Features überlegt, 
    welche dann als GitHub Issues festgehalten wurden. Falls wir uns nicht während des Meetings einigen konnten, wer was macht, 
    hat sich jeder selbst das zugewiesen, worauf er grade Lust hatte. Auch haben wir unseren Zwischenstand immer mal wieder unserem 
Auftraggeber Prof. Neugebauer gezeigt und uns Feedback dazu abgeholt. Somit konnten wir schnell sehen, ob etwas nicht so passt, 
    wie er es wollte. Die App selber haben wir in Java programmiert, was für mich gut war, da ich schon einige Erfahrung 
    in der Java Programmierung habe. Zwar habe ich noch nie eine Mobile App entwickelt, jedoch gewöhnte ich mich recht schnell 
    an die IDE und die neuen Bibliotheken. Programmiert habe ich immer nur alleine, jedoch habe ich mir auch ab und zu rat meiner 
    Teamkollegen eingeholt, bezüglich technischer Umsetzungen. Ein kleines Problem war noch, dass ich kein Android Handy besitze 
    und dementsprechend einige Features unserer App nicht ausprobieren konnte. Da wir untereinander jedoch gut vernetzt sind, 
    habe ich einfach mein neues Feature freigegeben und in unsere Gruppe geschrieben, ob es jemand testen könne. 
    Dort habe ich dann auch Feedback erhalten und konnte somit eventuelle Probleme direkt beheben. Alles in allem 
    war ich mit unserer Teamarbeit sehr zufrieden und bin mir sicher, dass wir ein gutes Produkt geschaffen haben.
\subsubsection{Aleksandr Pronin}
    Ich habe mein Team im Sommersemester kennengelernt. Zu diesem Zeitpunkt war die Android-App bereits teilweise fertig, 
    und wir mussten nur noch einen Plan für den Rest der Arbeit erstellen.
    Im ersten Schritt der Planung haben wir ein Verständnis geschaffen, wie wir die Rolle in unserem Team verteilen können. 
    Meine Aufgabe war es, den Client-Server-Teil der Anwendung zu entwickeln und die entsprechende Dokumentation zu schreiben. 
    Das Domänenmodell, das zeigt, welche Elemente im initialen MVP vorkommen sind, wurde schon aufgestellt.
    Und ich hatte also eine gute Vorstellung davon, wie meine Arbeit ablaufen sollte.  Im Rahmen agile Vorgehensweisen 
    sprachen wir ständig mit unserem Stakeholder Prof. Neugebauer, um die Anforderungen an die App zu 
    konkretisieren und optimale Lösung konzipieren zu können. Mit all diesen Informationen fingen wir an, technische Aspekte 
    und Entwürfe zu User Interfaces zu erarbeiten. \\
    Bei dem Projekt, das meine Kollegen unterstütz und mitentwickelt haben, handelt es sich um eine Android-App zur 
    Speicherung von GPS-Routen und zum Austausch dieser Daten zwischen Geräten über einen Server. Application Server wurde
    in Python unter Verwendung des Flask-Frameworks geschrieben und die clientseitigen Komponenten wurden von mir 
    in Java weiterentwickelt. Rückblickend kann ich sagen, dass wir im Rahmen der Gruppenarbeit  gute Verständnis von agile 
    Entwickeln der App  und insbesondere ein kompetentes Management von dem Team besser erfassen konnten. Ich kann auch bestätigen,
    dass wir Fähigkeiten sowohl bei der Ermittlung von Anforderungen als auch bei deren Spezifizierung erworben haben und alle
    Besonderheiten der Entwicklung komplexer Softwaresysteme erkannt haben. Aus meiner Erfahrung als Werkstudent kann ich mit Zuversicht sagen,
    dass wir gemeinsam hervorragende Ergebnisse erzielt haben. Es hat Spaß gemacht, erfolgreich mit dem Team zusammenzuarbeiten und das Projekt zu 
    unterstützen. Dank meiner Kollegen(oder auch Kommilitonen) und unseres Projekts konnte ich jede Menge lernen.
\subsubsection{Alex Schechtel}
    Im zweisemestrigen Modul Software Engineering wurde ich zum ersten mal damit konfrontiert, ein erstes großes Softwareprojekt in einem Team umzusetzen.
    Größtenteils war ich im Projekt in der Implementierung der App tätig und beschäftigte mich mit dem Overlay der App und den Funktionen verschiedener Buttons.
    Ich hatte zuvor keine Erfahrung in der Appentwicklung und ko    nnte viele neue Dinge lernen.
    Dazu gehören die Gestaltung einer App und Zuweisung von Funktionen an die Buttons,
    aber auch wie man ein komplexes Softwaresystem plant und am besten umsetzt. 
    Da wir Java als Programmiersprache der App benutzten und ich schon Vorkenntnisse in Java besaß, 
    war es kein Problem sich an die Appentwicklungsumgebung zu gewöhnen.\\ 
    Ein Problem, welches ich bei der Entwicklung der App hatte, war es, den implementierten Code der Teammitglieder zu verstehen.
    Aber da wir über eine sehr gute Kommunikation verfügten, wurden diese Probleme sehr schnell aus dem Weg geschafft.
    Wenn ich mal Fragen oder Probleme bei der Entwicklung hatte wurde mir von den Teamkollegen immer geholfen.
    In den Meetings (im Team) herschte immer ein sehr lockeres und angenehmes Arbeitsklima, 
    was zusätzlich zu einer guten Gruppenarbeit beigetragen hat. \\
    Die Kommunikation mit dem Themensteller bereitete uns auch keine Probleme. Es konnte immer nach Anfrage ein Termin gefunden werden. 
    Alles in allem hat mir das Projekt viel Spaß gemacht, da ich jede Menge neue Dinge lernen konnte und ich denke, 
    dass wir gute Arbeit geleistet haben.
\end{document}
