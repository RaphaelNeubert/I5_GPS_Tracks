\documentclass{article}

\usepackage[margin=3cm]{geometry}
\usepackage{amsmath}
\usepackage{amsfonts}
\usepackage{amssymb}
\usepackage{amscd}
\usepackage{standalone}
\usepackage{float}
\usepackage{color}
\usepackage[shortlabels]{enumitem}
\usepackage{graphicx}
\usepackage{caption}
\usepackage[ngerman]{babel}
\usepackage{lscape}
\usepackage{cancel}
\usepackage{dirtytalk}

\graphicspath{ {./images/} }

\begin{document}
\begin{titlepage}
    \centering
    {\scshape\LARGE Hochschule für Technik und Wirtschaft Dresden \par}
    \vspace{1cm}
    {\scshape\Large Softwaresystem \glqq GPS-Track-App\grqq\par}
    \vspace{1.5cm}
    {\huge\bfseries Prjektbericht\par}
    \vspace{2cm}
    {\Large\itshape Raphael Neubert, Alex Schechtel, Aleksandr Pronin, Quang Duy Pham, Tom Nicolai\par}
    \vfill

    {\large \today\par}
\end{titlepage}
\tableofcontents
\newpage
\section{Planung}
\section{Ergebnisse}
\subsection{Reflexion der Teammitglieder}
\subsubsection{Tom Nicolai}
\subsubsection{Aleksandr Pronin}
    Ich habe mein Team im Sommersemester kennengelernt. Zu diesem Zeitpunkt war die Android-App bereits teilweise fertig, 
    und wir mussten nur noch einen Plan für den Rest der Arbeit erstellen.
    Im ersten Schritt der Planung haben wir im Team ein Verständnis geschaffen, wie wir die Rolle in unserem Team verteilen können. 
    Meine Aufgabe war es, den Client-Server-Teil der Anwendung zu entwickeln und die entsprechende Dokumentation zu schreiben. 
    Das Domänenmodell, das uns zeigt, welche Elemente im initialen MVP vorkommen sind, wurde gut aufgestellt.
    Ich hatte also eine gute Vorstellung davon, wie meine Arbeit ablaufen sollte.  Im Rahmen agile Vorgehensweisen 
    sprachen wir ständig mit unserem Stakeholder Prof. Neugebauer, um die Ansprüche an die App zu konkretisieren. 
    Mit all diesen Informationen fingen wir an, technische Aspekte und Entwürfe zu User Interfaces zu erarbeiten. \\
    Bei dem Projekt, das meine Kollegen unterstütz und mitentwickelt haben, handelt es sich um eine Android-App zur 
    Speicherung von GPS-Routen und zum Austausch dieser Daten zwischen Geräten über einen Server. Application Server
    wurde in Python unter Verwendung des Flask-Frameworks geschrieben und die clientseitigen Komponenten wurden 
    in Java weiterentwickelt. Rückblickend kann ich sagen, dass wir im Rahmen der Gruppenarbeit Verständnis von agile 
    Entwickeln der App  und insbesondere ein kompetentes Management von dem Team besser erfassen konnten. Aufgrund meiner 
    Erfahrung als Werkstudent kann ich mit Sicherheit sagen, dass wir gemeinsam hervorragende Ergebnisse erzielt haben.
\end{document}
