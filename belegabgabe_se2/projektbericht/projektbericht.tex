\documentclass{article}

\usepackage[margin=3cm]{geometry}
\usepackage{amsmath}
\usepackage{amsfonts}
\usepackage{amssymb}
\usepackage{amscd}
\usepackage{standalone}
\usepackage{float}
\usepackage{color}
\usepackage[shortlabels]{enumitem}
\usepackage{graphicx}
\usepackage{caption}
\usepackage[ngerman]{babel}
\usepackage{lscape}
\usepackage{cancel}
\usepackage{dirtytalk}

\graphicspath{ {./images/} }

\begin{document}
\begin{titlepage}
    \centering
    {\scshape\LARGE Hochschule für Technik und Wirtschaft Dresden \par}
    \vspace{1cm}
    {\scshape\Large Softwaresystem \glqq GPS-Track-App\grqq\par}
    \vspace{1.5cm}
    {\huge\bfseries Prjektbericht\par}
    \vspace{2cm}
    {\Large\itshape Raphael Neubert, Alex Schechtel, Aleksandr Pronin, Quang Duy Pham, Tom Nicolai\par}
    \vfill

    {\large \today\par}
\end{titlepage}
\tableofcontents
\newpage
\section{Planung}
\section{Ergebnisse}
\subsection{Reflexion der Teammitglieder}
\subsubsection{Tom Nicolai}
    Ich war mit im Bereich der Programmierung der App zuständig. In den Meetings haben wir uns immer neue Features überlegt, 
    welche dann als GitHub Issues festgehalten wurden. Falls wir uns nicht während des Meetings einigen konnten, wer was macht, 
    hat sich jeder selbst das zugewiesen, worauf er grade Lust hatte. Auch haben wir unseren Zwischenstand immer mal wieder unserem 
    Auftraggeber Prof. Neugebauer gezeigt und uns Feedback dazu abgeholt. Somit konnten wir schnell sehen, ob etwas nicht so passt, 
    wie er es wollte. Die App selber haben wir in Java programmiert, was für mich gut war, da ich schon einige Erfahrung 
    in der Java Programmierung habe. Zwar habe ich noch nie eine Mobile App entwickelt, jedoch gewöhnte ich mich recht schnell 
    an die IDE und die neuen Bibliotheken. Programmiert habe ich immer nur alleine, jedoch habe ich mir auch ab und zu rat meiner 
    Teamkollegen eingeholt, bezüglich technischer Umsetzungen. Ein kleines Problem war noch, dass ich kein Android Handy besitze 
    und dementsprechend einige Features unserer App nicht ausprobieren konnte. Da wir untereinander jedoch gut vernetzt sind, 
    habe ich einfach mein neues Feature freigegeben und in unsere Gruppe geschrieben, ob es jemand testen könne. 
    Dort habe ich dann auch Feedback erhalten und konnte somit eventuelle Probleme direkt beheben. Alles in allem 
    war ich mit unserer Teamarbeit sehr zufrieden und bin mir sicher, dass wir ein gutes Produkt geschaffen haben.
\subsubsection{Aleksandr Pronin}
    Ich habe mein Team im Sommersemester kennengelernt. Zu diesem Zeitpunkt war die Android-App bereits teilweise fertig, 
    und wir mussten nur noch einen Plan für den Rest der Arbeit erstellen.
    Im ersten Schritt der Planung haben wir ein Verständnis geschaffen, wie wir die Rolle in unserem Team verteilen können. 
    Meine Aufgabe war es, den Client-Server-Teil der Anwendung zu entwickeln und die entsprechende Dokumentation zu schreiben. 
    Das Domänenmodell, das zeigt, welche Elemente im initialen MVP vorkommen sind, wurde schon aufgestellt.
    Und ich hatte also eine gute Vorstellung davon, wie meine Arbeit ablaufen sollte.  Im Rahmen agile Vorgehensweisen 
    sprachen wir ständig mit unserem Stakeholder Prof. Neugebauer, um die Anforderungen an die App zu 
    konkretisieren und optimale Lösung konzipieren zu können. Mit all diesen Informationen fingen wir an, technische Aspekte 
    und Entwürfe zu User Interfaces zu erarbeiten. \\
    Bei dem Projekt, das meine Kollegen unterstütz und mitentwickelt haben, handelt es sich um eine Android-App zur 
    Speicherung von GPS-Routen und zum Austausch dieser Daten zwischen Geräten über einen Server. Application Server
    wurde von mir in Python unter Verwendung des Flask-Frameworks geschrieben und die clientseitigen Komponenten wurden 
    in Java weiterentwickelt. Rückblickend kann ich sagen, dass wir im Rahmen der Gruppenarbeit Verständnis von agile 
    Entwickeln der App  und insbesondere ein kompetentes Management von dem Team besser erfassen konnten. Ich kann auch bestätigen,
    dass wir Fähigkeiten sowohl bei der Ermittlung von Anforderungen als auch bei deren Spezifizierung erworben haben und alle
    Besonderheiten der Entwicklung komplexer Softwaresysteme erkannt haben. Aufgrund meiner Erfahrung als Werkstudent kann ich mit 
    Sicherheit sagen, dass wir gemeinsam hervorragende Ergebnisse erzielt haben. Aus meiner Erfahrung als Werkstudent kann ich mit Zuversicht sagen,
    dass wir gemeinsam hervorragende Ergebnisse erzielt haben. Es hat Spaß gemacht, erfolgreich mit dem Team zusammenzuarbeiten und das Projekt zu 
    unterstützen. Dank meiner Kollegen(oder auch Kommilitonen) und unseres Projekts konnte ich viel lernen.
\end{document}
