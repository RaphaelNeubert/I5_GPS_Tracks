\documentclass{article}

\usepackage[margin=3cm]{geometry}
\usepackage{amsmath}
\usepackage{amsfonts}
\usepackage{amssymb}
\usepackage{amscd}
\usepackage{standalone}
\usepackage{float}
\usepackage{color}
\usepackage[shortlabels]{enumitem}
\usepackage{graphicx}
\usepackage{caption}
\usepackage[ngerman]{babel}
\usepackage{lscape}
\usepackage{cancel}
\usepackage{dirtytalk}
\usepackage{siunitx}
\usepackage{listings}

\graphicspath{ {./images/} }

\begin{document}
\begin{titlepage}
    \centering
    {\scshape\LARGE Hochschule für Technik und Wirtschaft Dresden \par}
    \vspace{1cm}
    {\scshape\Large Softwaresystem \glqq GPS-Track-App\grqq\par}
    \vspace{1.5cm}
    {\huge\bfseries Betriebsanleitung\par}
    \vspace{2cm}
    {\Large\itshape Raphael Neubert, Alexander Pronin \par}
    \vfill

    {\large \today\par}
\end{titlepage}

\tableofcontents
\newpage

\section{Server}
Der Server stellt eine REST-API zur Verfügung mit der sich GPS-Track auf dem Server 
speichern und herunterladen lassen.
\subsection{Allgemeiner Aufbau}
Der Server besteht aus einer einzigen Python-Datei.
Die GPS-Tracks werden in einen Ordner \say{files} gespeichert,
welcher sich (standartmäßig) im Pfad der Python-Datei befinden muss.
Der Ort an dem die GPS-Tracks gespeichert werden kann in der Python-Datei
angepasst werden. \\ \par
\textbf{Endpunkte:}
\begin{itemize}
    \item \textit{/} \\ Website zur Überprüfung, ob der Server online ist.
    \item \textit{/liste} \\ Ermöglicht es mithilfe von GET-Request eine Liste
        mit den, auf dem Server vorhandenen Dateien, zu erhalten.
    \item \textit{/dellist/}$<$\textit{path:filename}$>$ \\ Ermöglicht es 
        mithilfe von GET-Request eine Liste
        mit den, vom Server gelöschten Dateien, zu erhalten.
    \item \textit{/upload} \\  Ermöglicht es mithilfe von POST-Requests GPS-Tracks 
        auf den Server zu laden.
    \item \textit{/download/}$<$\textit{path:filename}$>$ \\  Ermöglicht es mithilfe
        von GET-Requests GPS-Tracks vom Server zu laden.
    \item \textit{/delete/}$<$\textit{path:filename}$>$  \\
        Ermöglicht es mithilfe von GET-Requests GPS-Tracks 
        vom Server zu löschen. Falls die Datei \say{./delete.txt} auf dem Server noch
        nicht existiert, wird diese erzeugt. Der Datei \say{./delete.txt} wird 
        der Name der durch den Request gelöschte Datei, angehangen.
\end{itemize}
\subsection{Anforderungen}
Der Server ist minimalistisch gehalten. Daher ist ein System mit wenig 
Leistung z.B. ein Raspberry Pi ausreichend. Es sollte sichergestellt 
werden, dass auf dem Server genügen Speicherplatz für die gewünschte Anzahl an 
GPS-Tracks vorhanden ist. Auf dem System muss ein Python3 Interpreter, sowie das 
Python-Framework Flask installiert sein.
\subsection{Installation}
\subsubsection{Debian}
Die folgenden Schritte gelten nicht nur für Debian selber, sondern auch für andere 
Linux-Distributionen die den Packetmanager \say{apt} verwenden. (z.B. Ubuntu, Mint)\par

\begin{enumerate}
    \item Installation der benötigten Programme: 
        \begin{itemize}
            \item Updaten des Betriebssytems sowie alle auf dem System 
                installierten Bibliotheken. \\
                \textit{sudo apt update} \\
                \textit{sudo apt upgrade}
            \item Installation von Python3 sowie den Python3-Packagemanger \say{pip3} \\
                \textit{sudo apt install python3} \\
                \textit{sudo apt install python3-pip}\\
                \textit{sudo apt install cron}
        \end{itemize}
    \item Installation der benötigten Python-Frameworks: 
        \begin{itemize}
            \item Installation von \say{Flask} mithilfe von \say{pip3} \\
                \textit{pip3 install flask}
        \end{itemize}
    \item Manuelles Starten des Servers: \\
        \textit{python3 app.py} 
    \item Automatisches Starten des Python-Programms beim Hochfahren des Servers: \\ 
        Kann auf verschieden realisiert werden (z.B. SystemD, OpenRC).
        Hier: mithilfe von \say{Cronjobs}.  \\
        \textit{crontab -e} \\ 
        Falls \say{Crontab} zum ersten mal ausgeführt wurde muss nun ein Editor gewählt
        werden. Anschließend muss folgende Zeile an das ende der nun offenen Datei 
        angehangen werden. \\
        \say{@reboot cd $<$path$>$ \&\& python3 app.py}
\end{enumerate}
\subsubsection{andere Betriebssysteme}
\begin{itemize}
    \item Die Installation auf andern Linux-basierten Betriebssystemen erfolgt
        fast identisch. Lediglich die Befehle zur Installation
        der Programme können anders sein.
    \item Die Installation auf Windows ist möglich und kann am einfachsten 
        unter Verwendung eines \say{Virtual Environments} realisiert werden. Dies 
        wird hier aber nicht weiter erläutert.
\end{itemize}
\subsection{Wartung}
Im Normalfall bedarf der Server keinerlei Wartung. 
Es sollte jedoch sichergestellt werden, dass das Betriebssystem regelmäßig aktualisiert
wird. Sollte nach langem Betrieb die Synchronisation lange dauern oder eine 
hohe Netzauslastung verursachen, kann es daran liegen, dass auf dem Server
die Datei \say{./delete.txt} zu groß geworden ist. Wenn alle Endgeräte auf dem 
gleichen Synchronisationsstand sind, kann diese Problemlos gelöscht werden.
\section{App}
\subsection{Building}
\subsection{Installation}
\subsubsection{per APK}
\subsubsection{über Android Studio}
\end{document}
