\documentclass{article}

\usepackage[margin=3cm]{geometry}
\usepackage{amsmath}
\usepackage{amsfonts}
\usepackage{amssymb}
\usepackage{amscd}
\usepackage{standalone}
\usepackage{float}
\usepackage{color}
\usepackage[shortlabels]{enumitem}
\usepackage{graphicx}
\usepackage{caption}
\usepackage[ngerman]{babel}
\usepackage{lscape}
\usepackage{cancel}
\usepackage{dirtytalk}
\usepackage{siunitx}
\usepackage{listings}

\graphicspath{ {./images/} }

\begin{document}
\begin{titlepage}
    \centering
    {\scshape\LARGE Hochschule für Technik und Wirtschaft Dresden \par}
    \vspace{1cm}
    {\scshape\Large Softwaresystem \glqq GPS-Track-App\grqq\par}
    \vspace{1.5cm}
    {\huge\bfseries Betriebsanleitung\par}
    \vspace{2cm}
    {\Large\itshape Raphael Neubert, Alexander Pronin \par}
    \vfill

    {\large \today\par}
\end{titlepage}

\tableofcontents
\newpage

\section{Server}
\subsection{Allgemeiner Aufbau}
Der Server besteht aus einer einzigen Python-Datei.
Die GPS-Tracks werden in einen Ordner \say{files} gespeichert,
welcher sich (standartmäßig) im Pfad der Python-Datei befinden muss.
Der Ort an dem die GPS-Tracks gespeichert werden kann in der Python-Datei
angepasst werden.
\subsection{Installation}
\subsubsection{Debian}
\subsubsection{andere Betriebssysteme}
\subsection{Wartung}
\section{App}
\subsection{Building}
\subsection{Installation}
\subsubsection{per APK}
\subsubsection{über Android Studio}
\end{document}
