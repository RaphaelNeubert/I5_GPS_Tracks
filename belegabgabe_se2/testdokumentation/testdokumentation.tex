\documentclass{article}

\usepackage[margin=3cm]{geometry}
\usepackage{amsmath}
\usepackage{amsfonts}
\usepackage{amssymb}
\usepackage{amscd}
\usepackage{standalone}
\usepackage{float}
\usepackage{color}
\usepackage[shortlabels]{enumitem}
\usepackage{graphicx}
\usepackage{caption}
\usepackage[ngerman]{babel}
\usepackage{lscape}
\usepackage{cancel}
\usepackage{dirtytalk}

\graphicspath{ {./images/} }

\begin{document}
\begin{titlepage}
    \centering
    {\scshape\LARGE Hochschule für Technik und Wirtschaft Dresden \par}
    \vspace{1cm}
    {\scshape\Large Softwaresystem \glqq GPS-Track-App\grqq\par}
    \vspace{1.5cm}
    {\huge\bfseries Testdokumentation\par}
    \vspace{2cm}
    {\Large\itshape Quang Duy Pham\par}
    \vfill

    {\large \today\par}
\end{titlepage}
\tableofcontents
\newpage

\section{Testkonzept}
\subsection{Testobjekte}
	In unserem Projekt müssen vier Hauptobjekte getestet werden:
	\begin{itemize}
		\item Hauptfunktionalität: Aufnahme eines Tracks.
		\item Hauptfunktionalität: Bearbeiten eines Tracks.
		\item Hauptfunktionalität: Synchronisation.
		\item Manuelle Tests für das Backend - Server.
		\item Nebenfunktionalitäten.
	\end{itemize}

\subsection{Testmethode}
	In unserem Projekt haben wir uns aus zwei gründen für Black-Box-Tests entschieden. Zunächst wird beim Black-Box-Testen die Funktionalität einer Anwendung untersucht, ohne auf ihre interne Struktur oder Funktionsweise zu achten. Zweitens kann dieser Ansatz auf alle Teststufen angewendet werden. Unsere App benötigt keine automatisierten Tests, daher verwenden wir manuelle Tests. \par
\subsection{Test Durchführungsplanung}
	Wir bevorzugen es, unsere Tests als End-to-End-Test zu planen, bei denen der Workflow einer Anwendung von Anfang bis Ende getestet wird. Das Verfahren soll reale Benutzerszenarien simulieren, um die Integration und Datenintegrität des Systems zu überprüfen. Der Testplan erfolgt in folgenden Schritten: \par
	\begin{enumerate}
		\item Anforderungen analysieren.
		\item Testumgebung einrichten.
		\item Auflisten, wie jedes System reagieren muss.
		\item Testfälle entwerfen.
		\item Tests durchführen.
	\end{enumerate}

\section{Testfallbeschreibung}
\subsection{Hauptfunktionalität: Aufnahme eines Tracks}
\subsubsection{Vorbedingungen}
	Die Aufnahme eines Tracks wird von folgenden Faktoren entschieden:
	\begin{itemize}
		\item Aufnahme läuft nicht nur reibunglos im Vorgrund sondern auch im Hintergrund, wenn das Handy im Standby Modus ist.
		\item Der aufgenomme Track wird mit einem vom Benutzer eingegeben Namen oder automatisch generierten Namen gespeichert.
		\item Durch drücken des Buttons \glqq Fortsetzen\grqq\space wird die Aufnahme normalweise weiter aufgenommen.
		\item Mit dem Button \glqq Verwerfen\grqq\space wird die Aufnahme gestoppt und die Änderungen verworfen.
	\end{itemize}
\subsubsection{Testdaten}
	Die Testdaten liefert uns das GPS-Signal des Handys. Die GPS Koordinaten werden als Punkte und diese als zusammenhängende Route auf der Karte angezeigt.
\subsubsection{Ablaufbeschreibung}
	Um die Funktionalität \glqq Aufnahme eines Tracks\grqq\space zu testen, läuft der Test wiefolgt:
	\begin{enumerate}
		\item Es wird jeweils eine Aufnahme im im Vordergrund und Hintergrund gemacht. Dabei wird geprüft, ob der Track generiert wird.
		\item Nun werden alle Buttons einmal betätigt:
			\begin{itemize}
				\item Speichern: prüfen, ob der aufgenommene Track richtig gespeichert wird.
				\item Fortsetzen: prüfen, ob die Aufnahme weiterer Punkte erfolgt.
				\item Verwerfen: prüfen, ob die Aufnahme verworfen wird.
			\end{itemize}
	\end{enumerate}
\subsubsection{Erwartetes Ergebnis}
	Bei Speichern, sollte der Track richtig gespeichert werden, bei Fortsetzen soll die Aufnahme fortgesetzt werden und bei Verwerfen soll der aufgenommene Track gelöscht werden und nicht gespeichert.

\subsection{Hauptfunktionalität: Bearbeiten eines Tracks}
\subsubsection{Vorbedingungen}
	Die Vorbedingungen zum bearbeiten eines Tracks sind:
	\begin{itemize}
		\item GPS-Punkte können auf anderne Standorte verschoben werden.
		\item Punkte können eine Beschreibung bekommen (für weitere Entwicklung mit dem Roboter).
		\item Der bearbeitete Track wird mit bisherigem Name automatisch gespeichert.
		\item Nach drücken des Buttons \glqq Fortsetzen\grqq\space wird weiter fortgesetzt.
		\item Mit dem Button \glqq Verwerfen\grqq\space wird das Bearbeiten gestoppt und die Änderungen werden verworfen.		
	\end{itemize}
\subsubsection{Testdaten}
	Testdaten für diesen Testobjekt sind neue Koordinaten der GPS-Punkte und Information des speziellen Markers.
\subsubsection{Ablaufbeschreibung}
	Der Ablauf des Test für die Funktionalität \glqq Bearbeiten eines Tracks\grqq ist:
	\begin{enumerate}
		\item Ein Punkt wird gedrücktgehalten und dann verschoben.
		\item Ein Punkt wird ausgewählt und eine Beschreibung hinzugefügt.
		\item Nun werden alle Buttons einmal betätigt:
			\begin{itemize}
				\item Speichern: prüfen, ob der bearbeitete Track richtig gespeichert wird.
				\item Fortsetzen: prüfen, ob die Bearbeitung weitererhin erfolgen kann.
				\item Verwerfen: prüfen, ob die Änderungen verworfen werden.
			\end{itemize}
	\end{enumerate}
\subsubsection{Erwartetes Ergebnis}
	Der Verschobene Punkt soll seine neue Position behalten und weiterhin mit den Anderen Punkten verbunden sein. Außerdem sollen alle Buttons ihren Beschreibungen entsprechend funktionieren.

\subsection{Hauptfunktionalität: Synchronisation}
\subsubsection{Vorbedigungen}
	FÜr die Synchronisation müssen folgende Bedigungen erfüllt sein:
	\begin{itemize}
		\item Tracks, die auf Server gespeichert sind, müssen auf Handy heruntergeladen werden.
	\end{itemize}
\subsubsection{Testdaten}
	Für die Synchronisation sind die Testdaten als Tracks auf dem Server.
\subsubsection{Ablaufbeschreibung}
	\begin{enumerate}
		\item Handy wird mit dem Internet und der VPN verbunden.
		\item Der Button \glqq Sync\grqq\space wird gedruckt.
		\item Prüfen, ob die neuen Tracks vom Server auf das Handy heruntergeladen werden.
	\end{enumerate}
\subsubsection{Erwartetes Ergebnis}
	Alle Tracks, die neu auf dem Server sind, sollten auf dem Handy gespeichert sein.

\subsection{Manuelle Tests für das Backend - Server}
\subsection{Vorbedingungen}
	Der Server sollte diese Anforderungen erfüllen:
		\begin{itemize}
			\item Der Server Online.
			\item Der Server sollte 95\% der Laufzeit verfügbar sein und den Clients antworten
		\end{itemize}
\subsubsection{Testdaten}
	Es gibt keine Testdaten in diesem Test-Case, weil wir nur die Funktionalitaet des Servers testen.
\subsubsection{Ablaufbeschreibung}
	Mit dem Kommandozeilenprogramm "ping" prüfen, ob der Server überhaupt antwortet. Ebenso kann man über das Terminal das Programm "curl" verwenden: curl -i http://141.56.137.84:5000/
\subsection{Erwartetes Ergebnis}
	Erwartet wird, dass der Server mit \glqq OK\grqq\space und dem Code 200 antwortet.
	
\subsection{Nebenfunktionalitäten}
\subsubsection{Vorbedingungen}
	In der App gibt es fünf Nebenfunktionalitäten, für welche die folgenden Bedingungen gelten:
	\begin{itemize}
		\item Details eines GPS-Tracks anzeigen
		\item GPS-Track auf Karte anzeigen	
		\item Aufnahme fortsetzen
		\item GPS-Track umbenennen
		\item GPS-Track löschen	
	\end{itemize}
\subsubsection{Testdaten}
	Alle gelistete und gespeicherte Tracks sind Testdaten.
\subsubsection{Ablaufbeschreibung}
	Nebenfunktionalitäten werden gewählt und geprüft, ob
		\begin{itemize}
			\item die Informationen des GPS-Tracks angezeigt werden. z.B.: Die Zeit der Aufnahme und die Länge des Tracks.
			\item der gespeicherte Track als verbundene Punkte auf Karte angezeigt wird.
			\item der gespeicherte Track weiter aufgenommen wird.
			\item der gespeicherte Track umbenannt werden kann.
			\item der gespeicherte Track komplett gelöscht wird.
		\end{itemize}
\subsubsection{Erwartetes Ergebnis}
	Alle Nebenfunktionalität sollten erfolgreich funktionieren und mit anderen Funktionalitäten gut integiert werden.

\section{Ergebnisse}
\subsection{Ergebnisse der Testdurchführung}
	Nachdem alle Tests abgeschlossen sind, sind alle funktionalen Anforderungen erfüllt und gut in die Anwendung integriert. Die App hat jedoch immer noch einige Fehler und Bugs wie:
	\begin{itemize}
		\item Wenn der Benutzer \glqq GPS-Track umbenennen\grqq\space oder \glqq SYNC\grqq\space drückt, können andere Buttons noch gedrückt werden, jedoch funktionieren sie nicht so wie sie sollten.
		\item Bei Installation der App werden keine Berechtigungen abgefragt.
		\item Bei Nutzung der Funktionalität \glqq Aufnahme fortsetzen\grqq\space kann es in seltenen fällen zu abstürzen kommen.
	\end{itemize}
	Alle Fehler und Bugs sind bei Abnahme behoben und erneut geprüft.
\subsection{Konsequenzen der erkannten Abweichungen}
	Die Abweichungen können zu einer schlechten Benutzererfahrung führen. Darüber hinaus können andere Funktionen der App falsch ausgeführt werden und das gesamte System kann zum Absturz gebracht werden. Im schlimmsten Fall können Benutzer ihre Daten der App nicht wiederherstellen.
\end{document}
