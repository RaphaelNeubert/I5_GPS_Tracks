\documentclass{article}

\usepackage[margin=3cm]{geometry}
\usepackage{amsmath}
\usepackage{amsfonts}
\usepackage{amssymb}
\usepackage{amscd}
\usepackage{standalone}
\usepackage{float}
\usepackage{color}
\usepackage[shortlabels]{enumitem}
\usepackage{graphicx}
\usepackage{caption}
\usepackage[ngerman]{babel}
\usepackage{lscape}
\usepackage{cancel}
\usepackage{dirtytalk}
\usepackage{siunitx}
\usepackage{listings}

\graphicspath{ {./images/} }

\begin{document}
\begin{titlepage}
    \centering
    {\scshape\LARGE Hochschule für Technik und Wirtschaft Dresden \par}
    \vspace{1cm}
    {\scshape\Large Softwaresystem \glqq GPS-Track-App\grqq\par}
    \vspace{1.5cm}
    {\huge\bfseries Nutzeranleitung\par}
    \vspace{2cm}
    {\Large\itshape Raphael Neubert... \par}
    \vfill

    {\large \today\par}
\end{titlepage}

\tableofcontents
\newpage

\section{Aufnahme eines GPS-Tracks}
\section{Bearbeiten eines GPS-Tracks}
\subsection{Hinzufügen von Special Points of Interest}
\section{GPS-Tracks mit Server synchronisieren}
\section{Nebenfunktionalitäten}
\subsection{GPS-Track auf Karte anzeigen}
\subsection{Details eines GPS-Tracks anzeigen}
\subsection{Aufnahme fortsetzen}
\subsection{GPS-Track umbenennen}
\subsection{GPS-Track löschen}
\end{document}
